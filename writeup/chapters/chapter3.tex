%! TEX root = /home/hsartoris/sproj/writeup/main.tex
\graphicspath{ {resources/models/3neurEx/9/} {resources/models/3neurEx/weights/} 
} 

\chapter{Results}
\label{results}
\section{3-neuron generator}
\label{results_3neur}
We first consider a generator network consisting of three nodes connected as in 
FIGURE HERE (should contain visual structure and adjacency matrix). All weights 
are binary, and a spike rate of .25 was used.\footnote{SEE APPENDIX	for 
information on spike rates}

\begin{table}[h]
	\centering
	
\begin{tikzpicture}[baseline=(current bounding box.center),->,>=stealth', 
	node distance=5em, semithick]
	\tikzstyle{every state}=[fill=none, draw=black, text=black]

	\node[state] (0) {0};
	\node[state] (1) [right of=0] {1};
	\node[state] (2) [below right of=0] {2};

	\path 	(0) edge node {} (1)
			(0) edge node {} (2)
			(1) edge node {} (2);
\end{tikzpicture}

	\hspace{2em}
	\begin{tabular}{l|lll}
		  & 0 & 1 & 2\\
		\hline
		0 & 0 & 0 & 0\\
		1 & 1 & 0 & 0\\
		2 & 1 & 1 & 0
	\end{tabular}
	\captionof{figure}{Network structure and adjacency matrix of the generator.}
	\label{fig:2simplex+adjacency}
\end{table}

While reconstructing a graph comprising only three nodes is not much of a feat, 
this simplified case allows us to demonstrate that our convolutional approach is 
capable of reconstruction at all. Furthermore, the small network size requires 
few timesteps and a small interlayer featurespace; i.e., $b,d<10$. This results 
in a relatively simple set of transitions, allowing us to explore and understand 
the inner workings of the network.

\subsection{Example Model}
\label{subsec:3neurex}
The following data are pulled from a model trained on data produced by the 
generator in figure \ref{fig:2simplex+adjacency}. Figure 
\ref{fig:3neur_loss+params} demonstrates the model's loss over time. In this 
example, \textit{b} and \textit{d} were pushed down in order to allow for better 
comprehension of the internal mechanics; the loss tends to converge more 
effectively and evenly given more computation power.
%! TEX root = /home/hsartoris/sproj/writeup/main.tex
\begin{table}[h]
	\centering
	\begin{minipage}{.48\textwidth}
	\resizebox{\textwidth}{!}{
		\begin{tikzpicture}
			\begin{semilogyaxis} [xlabel=Step, ylabel=Loss, scaled x 
				ticks=false,
				axis lines*=left,
				xtick={1,25000,50000},
				legend cell align=left,
				legend style={ anchor=north east},
				extra y ticks={.0497}, extra y tick style={grid=major},
				%ytick={1}, 
				yticklabel style={	/pgf/number format/precision=2,
									/pgf/number format/fixed}]
				\addplot [color=black] table [x=Step, y=Loss, col sep=comma, 
			mark=none, smooth] {../resources/models/9neur/minFinalConv_90};
				\addlegendentry{Locality-based}
				\addplot [color=black, dashed] table [x=Step, y=Loss, col 
			sep=comma, mark=none, smooth] 
			{../resources/models/9neur/minFinalDumb_72};
				\addlegendentry{Benchmark}
			\end{semilogyaxis}
		\end{tikzpicture}
	}
	\end{minipage}
	\hfill
	\begin{minipage}{.48\textwidth}
		\centering
		\begin{tabular}{lr}
			\textit{b} (timesteps) & 30\\
			\textit{d}& 40\\
			Batch size& 32\\
			Learning rate& .001\\
			Training samples& 36000\\
			Validation samples& 9000\\
		\end{tabular}
	\end{minipage}
	\captionof{figure}{Loss \& parameters for model trained on data from 
		generator given in \figref{fig:10neur}}
	\label{fig:9neur_loss+params}
\end{table}


\subsubsection{Trained Network Operation}
Here, we examine in brief the internal operation of the trained model over a 
single input. For a complete look through the procedure of reconstruction for 
this network, please see APPENDIX.
\begin{figure}[h]
	\centering
	\begin{subfigure}{.15\textwidth}
		\centering
		\includegraphics[width=.75\textwidth]{input.png}
		\caption{Input}
	\end{subfigure}
	\hspace{1em}
	\begin{subfigure}{.3\textwidth}
		\includegraphics[width=\textwidth]{out0.png}
		\caption{Data after first transform}
	\end{subfigure}
	\hspace{1em}
	\begin{subfigure}{.3\textwidth}
		\includegraphics[width=\textwidth]{out1.png}
		\caption{Data after convolutional layer}
		\label{subfig:3neur_out1}
	\end{subfigure}
	\caption{Path of data through network, up to final transform}
	\label{fig:3neur_input}
\end{figure}

The last transformation of the network involves a matrix multiplication of the 
final layer weights\footnote{see appendix} with the data in 
\ref{subfig:3neur_out1}. This produces an adjacency matrix matching that of 
\ref{fig:2simplex+adjacency}.

\section{Applicability Beyond Training Data}
As described in \ref{subsubsec:hotswap}, the fact that our model is trained on 
data produced by only one generator is of little consequence; due to its 
structure, the only information it can learn is relational; i.e., 
per-neuron-pair. Consider the following examples, in which data was produced 
from several generator networks and fed into the model described in 
\ref{results_3neur}:
