%! TEX root = /home/hsartoris/sproj/writeup/main.tex

\chapter{Introduction}

Reconstruction of biological neural networks given lossy imaging data of regions 
in question remains a persisent problem in the field of computational 
neuroscience, with artifical neural network-based solutions emerging in recent 
years as a preeminent method for achieving accurate 
reconstructions.\cite{Ray2015} However, the methods used are often abstruse and 
inaccessible, reducing accessibility; beyond this, there are features unique to 
biological neural networks that can be taken advantage of in producing 
reconstructions. We present an architecture for determining network structure 
inspired by convolutional neural networks. Whereas in image processing, the 
typical target of convolutional networks, pixel and feature adjacency correlates 
with shared meaning, there exists no such metric for data extracted from 
biological neural networks, as per-neuron spike trains can be reconfigured into 
various permutations without necessitating a change in the structure of the 
network that generated those spikes. Thus our architecture redefines the 
adjacency usually used in convolution to one more suited to the unique features 
of biological neural networks, derived from locality within the original graph 
structure.
