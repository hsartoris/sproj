%! TEX root = /home/hsartoris/sproj/writeup/main.tex
\chapter{Model}
\label{model}
The model trained and tested here represents ... stuff

\section{Data}
\label{sec:data}
Insofar as we treat ANNs as providing arbitrary function approximation, training
a network requires input data representing the known data about the system we
wish to model, as well as output data we wish the network to produce from the
inputs. More generally, input data usually entails information that is easy to 
acquire about the process being modeled, while output data, or labels, 
correspond to a dataset that is difficult to acquire generally. Of course, this 
means that the first step in training a neural network is to assemble a 
sufficiently large set of inputs and outputs in order to fully, or at least 
approximately, characterize the problem at hand.

In our case, we wish to map from (relatively) easily available data about 
biological networks, individual neuron spike times, to network structure. While 
such data exist, generating our own allows us to better analyze the results of 
the algorithm.


\subsection{Generation}
In order to demonstrate the validity of our algorithm for graph convolution, we 
opt for a simplified form of the kind of data that would be used in a real-world 
setting.  To this end, we create adjacency matrices representing simple, 
small-\textit{n} toy networks.

FIG: 3 neuron model \& associated adjacency matrix

Binary values are used throughout these toy networks: either a connection exists 
or it doesn't; either a `neuron' is spiking or it isn't. To produce spiking 
data, we create an \textit{n}-vector $\mathbb{S}$ representing the current state 
of the toy network, with random neurons already spiking based on a chosen spike 
rate. From here, the process is as follows, where $\mathbb{M}$ is the adjacency 
matrix:

\[
	\underset{n \times n}{\mathbb{M}} \times \underset{n \times 1}{\mathbb{S}^t} 
	= \underset{n \times 1}{\mathbb{S}^{t+1}}
\]
Additonally, $\mathbb{S}^{t+1}$ may have one or more neurons spike randomly, as 
determined by the spike rate of the simulation.\footnote{SEE APPENDIX} All 
values are clipped to the range $[0,1]$, to avoid double spiking.

At each step, $\mathbb{S}$ is appended to an output matrix, which is saved after 
simulation is complete. For $t$ simulation steps, the completed output has shape 
$(n \times t)$.

APPENDIX BIT: have a section on how spike rates were determined for each 
network, as well as an example of producing data for the 3-neuron case.

\subsubsection{Generalizability}
\label{subsubsec:hotswap}
In most ANN implementations, feeding various data with the same label attached 
to it results in the network learning to ignore the input data and always spit 
out the desired label, rendering it useless. However, due to the unique 
structure of our model, this sort of overfitting is impossible (SEE SOME 
ARCHITECTURE SECTION). Therefore, we must merely construct a suitably 
representative generator network, meaning that it contains all of the 
inter-neuron relationships we expect to see in the data we ultimately feed in to 
test.

\subsection{Restructuring}
The model accepts data in the form of a spike-time raster plot of dimensions $(n 
\times t)$, where \textit{n} is the number of neurons and \textit{t} is the 
number of timesteps being considered. The axes are reversed in comparison to the 
data created by the generator, and thus in the process of loading in the spike 
trains we transpose the matrices to the expected dimensionality. Additionally, 
it is not always necessary to use the full number of steps generated, depending 
on the size of the generator network in question, as well as its spike rate. In 
such a scenario, we truncate the time dimension appropriately.

EXAMPLE HERE
