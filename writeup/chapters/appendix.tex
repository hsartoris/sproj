%! TEX root = /home/hsartoris/sproj/writeup/main.tex
\begin{appendices}
	\chapter{Parameter Optimization Miscellanea}
	\section{Data}
	\subsection{Spike Rate Determination}
	As seen in section \ref{subsec:cyclical}, oversaturated data hamstrings the 
	ability of our model to perform accurate reconstructions. As the 

	\chapter{Model}
	\section{Batched Architecture Calculations}
	\label{asec:batched}
	In order to allow processing of many pieces of data at once, the matrix 
	model defined in \ref{subsec:matmodel} was adapted to a batched format.  
	Given input matrices of shape $(b \times n)$, the actual input to the model 
	is now of shape $(batchSize \times b \times n)$. As previously discussed, 
	iteration across lists or dimensions is not a computationally efficient 
	option. Therefore we use \texttt{tf.einsum}, an implementation of Einstein 
	Sums. This allows, for example, the multiplication of two matrices, one of 
	dimension $(i \times j \times k)$, and the other of dimension $(h \times 
	j)$. An appropriate function call might appear as
	\texttt{tf.einsum(`hj,ijk->ihk', mat2, mat1)}. The result is equivalent to 
	the iterative multiplication of the $(h \times j)$ matrix across all 
	\textit{i}, without the computational overhead of CPU involvement. Every 
	matrix multiplication in our model is implemented using this functionality.


\end{appendices}
